\documentclass[12pt,letterpaper]{article}
\usepackage{graphicx,textcomp}
\usepackage{natbib}
\usepackage{setspace}
\usepackage{fullpage}
\usepackage{color}
\usepackage[reqno]{amsmath}
\usepackage{amsthm}
\usepackage{fancyvrb}
\usepackage{amssymb,enumerate}
\usepackage[all]{xy}
\usepackage{endnotes}
\usepackage{lscape}
\newtheorem{com}{Comment}
\usepackage{float}
\usepackage{hyperref}
\newtheorem{lem} {Lemma}
\newtheorem{prop}{Proposition}
\newtheorem{thm}{Theorem}
\newtheorem{defn}{Definition}
\newtheorem{cor}{Corollary}
\newtheorem{obs}{Observation}
\usepackage[compact]{titlesec}
\usepackage{dcolumn}
\usepackage{tikz}
\usetikzlibrary{arrows}
\usepackage{multirow}
\usepackage{xcolor}
\newcolumntype{.}{D{.}{.}{-1}}
\newcolumntype{d}[1]{D{.}{.}{#1}}
\definecolor{light-gray}{gray}{0.65}
\usepackage{url}
\usepackage{listings}
\usepackage{color}


\definecolor{codegreen}{rgb}{0,0.6,0}
\definecolor{codegray}{rgb}{0.5,0.5,0.5}
\definecolor{codepurple}{rgb}{0.58,0,0.82}
\definecolor{backcolour}{rgb}{0.95,0.95,0.92}

\lstdefinestyle{mystyle}{
	backgroundcolor=\color{backcolour},   
	commentstyle=\color{codegreen},
	keywordstyle=\color{magenta},
	numberstyle=\tiny\color{codegray},
	stringstyle=\color{codepurple},
	basicstyle=\footnotesize,
	breakatwhitespace=false,         
	breaklines=true,                 
	captionpos=b,                    
	keepspaces=true,                 
	numbers=left,                    
	numbersep=5pt,                  
	showspaces=false,                
	showstringspaces=false,
	showtabs=false,                  
	tabsize=2
}
\lstset{style=mystyle}
\newcommand{\Sref}[1]{Section~\ref{#1}}
\newtheorem{hyp}{Hypothesis}

\title{Problem Set 2}
\date{Due: February 4, 2026}
\author{Data Visualization\\
	Shelly Veal-Upham\\
	25337422}
	
\begin{document}
	\maketitle
	
	\section*{Instructions}
	\begin{itemize}
	\item Please show your work! You may lose points by simply writing in the answer. If the problem requires you to execute commands in \texttt{R}, please include the code you used to get your answers. Please also include the \texttt{.R} file that contains your code. If you are not sure if work needs to be shown for a particular problem, please ask.
\item Your homework should be submitted electronically on GitHub.
\item This problem set is due before 23:59 on Wednesday February 4, 2026. No late assignments will be accepted.
	\end{itemize}
	
	\vspace{.25cm}
	\section*{Study of Religious Congregations in Switzerland}
	
The data for this problem set come from the	National Congregations Study Switzerland (NCSS), which was conducted in 2008–2009 and 2022–2023. The data provide information on organisational structure, staffing, finances, worship practices, youth and educational activities, social composition, external engagement, and inclusion norms. The data were collected using stratified random samples of congregations drawn from comprehensive censuses, with interviews completed by a single knowledgeable key informant in each congregation, most often the spiritual leader.

\subsection*{Data Manipulation}

\begin{enumerate}
\item Load the NCSS .csv file from \href{https://raw.githubusercontent.com/ASDS-TCD/DataViz_2026/refs/heads/main/datasets/NCSS_v1.csv}{GitHub} into your global environment. Use the select() function to keep these variables in your dataframe:
\begin{itemize}
	\item Congregation ID (\texttt{CASEID})
	\item Year (\texttt{YEAR})
	\item Region (\texttt{GDREGION})
	\item Number of official members (\texttt{NUMOFFMBR})
	\item 6-level religious classification (\texttt{TRAD6})
	\item 12-level religious classification (\texttt{TRAD12})
	\item Total income in last fiscal year (\texttt{INCOME})
\end{itemize}
	
	\lstinputlisting[language=R, firstline=42,lastline=54]{PS02_SVU.R}
	
\item Filter the dataset so that you only include Christian, Jewish, and Muslim congregations (Chrétiennes, Juives, Musulmanes) using the \texttt{TRAD6} variable.

	* Note here that the r code reflects "Chretiennes" whereas the dataset uses Chrétiennes -- I did actually filter with the use of Chrétiennes, but for compiling purposes it is reflected here as "Chretiennes."

	\lstinputlisting[language=R, firstline=58,lastline=63]{PS02_SVU.R}

\item Compute for the number of congregations by religious classification (\texttt{TRAD6}) in each year, as well as the mean and median total income in last fiscal year (\texttt{INCOME}) by religious classification and year.

	\lstinputlisting[language=R, firstline=67,lastline=82]{PS02_SVU.R}

\item Create a categorical variable for called \texttt{AVG\_INCOME} that is binary in which 1 = "Above average or average income" and 0 = "Below average income", which indicates if a congregation is $\geq$ average income or $<$ average income among congregations that year.

	\lstinputlisting[language=R, firstline=86,lastline=101]{PS02_SVU.R}

\end{enumerate}

\subsection*{Data Visualization}

\begin{enumerate}
	\item Create a bar plot visualizing the proportion of congregations by 12-level religious classification (\texttt{TRAD12}) in each year.
	
	\lstinputlisting[language=R, firstline=110,lastline=134]{PS02_SVU.R}
	
	\begin{figure}[h!]
		\caption{Proportions of Congregations by Year}\label{fig1}
		\centering
		\includegraphics[width=.8\textwidth]{bar_plot_1.1.pdf}
	\end{figure}
	
	In Figure~\ref{fig1} we have the proportions of congregations wrapped by year (2009 \& 2022), but I thought it'd be helpful also to see the years side-by-side for each congregation so we can visualize their proportional growth (see Figure~\ref{fig2}).
	
	\lstinputlisting[language=R, firstline=138,lastline=152]{PS02_SVU.R}
	
	\begin{figure}[h!]
		\caption{Proportional Growth of Congregations from 2009 to 2022}\label{fig2}
		\centering
		\includegraphics[width=.95\textwidth]{bar_plot_1.2.pdf}
	\end{figure}

	\newpage

	Professor Ziegler, in the updated version of the problem set, requested a proportional bar plot that distinguishes those above and below average income. See Figure~\ref{fig3} for my version.


	\lstinputlisting[language=R, firstline=331,lastline=342]{PS02_SVU.R}
	
	\begin{figure}[h!]
		\caption{Proportion of Congregations Above and Below Average Income}\label{fig3}
		\centering
		\includegraphics[width=.95\textwidth]{prop_bar_stacked.pdf}
	\end{figure}
		
	\newpage
	
	
	\item Make a histogram detailing the number of official members using the 12-level religious classification (\texttt{TRAD12}) distinguishing between the 6-level religious classification (\texttt{TRAD6}) by year. Hint: Use \texttt{facet()} for year, \texttt{TRAD6} on the x-axis, and group/fill using \texttt{TRAD12} with the \texttt{position="dodge"}.
	
	\lstinputlisting[language=R, firstline=159,lastline=168]{PS02_SVU.R}
	\newpage
	
	\begin{figure}[h!]
		\caption{Number of Members by Congregation}\label{fig4}
		\centering
		\includegraphics[width=.8\textwidth]{bar_plot_2.1.pdf}
	\end{figure}
	
	For Figure~\ref{fig4} I have used \texttt{geom\_col()} to compile the total numbers of official members in each congregation and grouped the x-axis by the variable \texttt{TRAD6}. Figure~\ref{fig5} shows much the same information, but wrapped by overarching religious affiliation (\texttt{TRAD12}) with differing y-axis scales.
	
	\lstinputlisting[language=R, firstline=285,lastline=302]{PS02_SVU.R}
	\vspace{5cm}
	
	\begin{figure}[h!]
		\caption{Number of Congregation Members by Religion}\label{fig5}
		\centering
		\includegraphics[width=.95\textwidth]{bar_plot_2.2.pdf}
	\end{figure}
	
	\item Display the distribution of congregations in 2022 above and below the average yearly income (\texttt{AVG\_INCOME}) in each region using ridge plots.
	
	\lstinputlisting[language=R, firstline=174,lastline=189]{PS02_SVU.R}
	
	\begin{figure}[h!]
		\caption{Congregations Above and Below Average Yearly Income}\label{fig6}
		\centering
		\includegraphics[width=.9\textwidth]{ridges_together.pdf}
	\end{figure}
	
	\newpage
	
	Because the scale of income is so different for those congregations with above and below average yearly incomes, I decided also to separate the two groups into individual plots for a closer look. See Figures~\ref{fig7} and~\ref{fig8}.
	
	\lstinputlisting[language=R, firstline=223,lastline=236]{PS02_SVU.R}
	
	\lstinputlisting[language=R, firstline=241,lastline=254]{PS02_SVU.R}
	
	\begin{figure}[h!]
		\caption{Congregations Above Average Yearly Income}\label{fig7}
		\centering
		\includegraphics[width=.95\textwidth]{ridges_above.pdf}
	\end{figure}
	
	
	\begin{figure}[h!]
		\caption{Congregations Below Average Yearly Income}\label{fig8}
		\centering
		\includegraphics[width=.95\textwidth]{ridges_below.pdf}
	\end{figure}
	
	
	\newpage
	
	\item Create a boxplot of the number of official members by year and region.
	
	\lstinputlisting[language=R, firstline=197,lastline=210]{PS02_SVU.R}
	
	\begin{figure}[h!]
		\caption{Official Members by Region}\label{fig9}
		\centering
		\includegraphics[width=.8\textwidth]{box_plot.pdf}
	\end{figure}

	

	Again, given the drastic difference in scales among religious affiliation membership, I decided to wrap the plot according to affiliation. See Figure~\ref{fig10}.
	
	\lstinputlisting[language=R, firstline=262,lastline=276]{PS02_SVU.R}
	
	\begin{figure}[h!]
		\caption{Regional Membership by Religious Affiliation}\label{fig10}
		\centering
		\includegraphics[width=.95\textwidth]{box_plot_wrapped.pdf}
	\end{figure}
	
	
	Finally, Professor Ziegler suggested in the updated form of the problem set that we wrap by region. Figure~\ref{fig11} shows what I did for that.

	\lstinputlisting[language=R, firstline=310,lastline=324]{PS02_SVU.R}

	\begin{figure}[h!]
		\caption{Regional Membership by Religious Affiliation}\label{fig11}
		\centering
		\includegraphics[width=.95\textwidth]{box_plot_by_reg.pdf}
	\end{figure}


\end{enumerate}


\end{document}
