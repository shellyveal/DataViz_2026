\documentclass[12pt,letterpaper]{article}
\usepackage{graphicx,textcomp}
\usepackage{natbib}
\usepackage{setspace}
\usepackage{fullpage}
\usepackage{color}
\usepackage[reqno]{amsmath}
\usepackage{amsthm}
\usepackage{fancyvrb}
\usepackage{amssymb,enumerate}
\usepackage[all]{xy}
\usepackage{endnotes}
\usepackage{lscape}
\newtheorem{com}{Comment}
\usepackage{float}
\usepackage{hyperref}
\newtheorem{lem} {Lemma}
\newtheorem{prop}{Proposition}
\newtheorem{thm}{Theorem}
\newtheorem{defn}{Definition}
\newtheorem{cor}{Corollary}
\newtheorem{obs}{Observation}
\usepackage[compact]{titlesec}
\usepackage{dcolumn}
\usepackage{tikz}
\usetikzlibrary{arrows}
\usepackage{multirow}
\usepackage{xcolor}
\newcolumntype{.}{D{.}{.}{-1}}
\newcolumntype{d}[1]{D{.}{.}{#1}}
\definecolor{light-gray}{gray}{0.65}
\usepackage{url}
\usepackage{listings}
\usepackage{color}

\definecolor{codegreen}{rgb}{0,0.6,0}
\definecolor{codegray}{rgb}{0.5,0.5,0.5}
\definecolor{codepurple}{rgb}{0.58,0,0.82}
\definecolor{backcolour}{rgb}{0.95,0.95,0.92}

\lstdefinestyle{mystyle}{
	backgroundcolor=\color{backcolour},   
	commentstyle=\color{codegreen},
	keywordstyle=\color{magenta},
	numberstyle=\tiny\color{codegray},
	stringstyle=\color{codepurple},
	basicstyle=\footnotesize,
	breakatwhitespace=false,         
	breaklines=true,                 
	captionpos=b,                    
	keepspaces=true,                 
	numbers=left,                    
	numbersep=5pt,                  
	showspaces=false,                
	showstringspaces=false,
	showtabs=false,                  
	tabsize=2
}
\lstset{style=mystyle}
\newcommand{\Sref}[1]{Section~\ref{#1}}
\newtheorem{hyp}{Hypothesis}

\title{Problem Set 3}
\date{Due: February 18, 2026}
\author{Data Visualization\\
	Shelly Veal-Upham\\
	25337422}

\begin{document}
	\maketitle
	
	\section*{Instructions}
	\begin{itemize}
	\item Please show your work! You may lose points by simply writing in the answer. If the problem requires you to execute commands in \texttt{R}, please include the code you used to get your answers. Please also include the \texttt{.R} file that contains your code. If you are not sure if work needs to be shown for a particular problem, please ask.
\item Your homework should be submitted electronically on GitHub.
\item This problem set is due before 23:59 on Wednesday February 18, 2026. No late assignments will be accepted.
	\end{itemize}
	
	\vspace{.25cm}
	\section*{Canadian Election Study}
	
The data for this problem set come from the	Canadian Election Study (\href{https://ces-eec.sites.olt.ubc.ca/files/2017/04/CES2015_Combined_Data_Codebook.pdf}{CES})  in 2015. The main purpose of the study is to give a comprehensive picture of the Canadian election: why people vote as they do, what changes during campaigns and across elections, and how Canadian voting compares with that in other democracies.

\subsection*{Data Manipulation}

\begin{enumerate}
\item Load the CES .csv file from \href{https://raw.githubusercontent.com/ASDS-TCD/DataViz_2026/refs/heads/main/datasets/CES2015.csv}{GitHub} into your global environment. Filter respondents to only include "high quality" participants: 
\begin{verbatim}
ces2015 <- ces2015 |> filter(discard == "Good quality")
\end{verbatim}

\item Filter the dataset to those participants that answered the question about voting for the past election using \texttt{p\_voted}. Consider respondents who gave a "Yes" answer as having voted, while “No” as not having voted. Treat “Don’t know” and “Refused” as missing.

	\lstinputlisting[language=R, firstline=98,lastline=101]{PS03_SVU.R}

\item Create an age variable and group into categories (e.g., $<$30, 30-44, 45-64, 65+). Year of birth is in age (four‑digit year).

	\lstinputlisting[language=R, firstline=105,lastline=113]{PS03_SVU.R}
	
\end{enumerate}

\subsection*{Data Visualization}

\begin{enumerate}
	\item Plot turnout rate by age group.
	
	\lstinputlisting[language=R, firstline=123, lastline=154]{PS03_SVU.R}
	
	\begin{figure}[h!]
		\caption{Turnout Percentages}\label{fig1}
		\centering
		\includegraphics[width=.9\textwidth]{1.pdf}
	\end{figure}
	
	\item Create a density plot of ideology by party, restricting your sample to respondents with non‑missing left–right self‑placement (0–10 scale) and those that intended to vote for a main party (e.g., Liberal, Conservative, NDP, Bloc in Quebec, and Green).
	
	
	\lstinputlisting[language=R, firstline=162, lastline=186]{PS03_SVU.R}
	
	\begin{figure}[h!]
		\caption{Self-Placement on Left-Right Political Scale by Region}\label{fig2}
		\centering
		\includegraphics[width=.9\textwidth]{2.pdf}
	\end{figure}
	
	\newpage
	
	\item Produce histogram counts of turnout by income (\texttt{income\_full}), faceted by province.
	
	Data wrangling:
	
	\lstinputlisting[language=R, firstline=190, lastline=209]{PS03_SVU.R}
	
	Plot 1:
	
	\lstinputlisting[language=R, firstline=211, lastline=234]{PS03_SVU.R}
	\lstinputlisting[language=R, firstline=309, lastline=311]{PS03_SVU.R}
	
	\begin{figure}[h!]
		\caption{Voting Turnout Lower Population Provinces, NA's included}\label{fig3}
		\centering
		\includegraphics[width=1\textwidth]{3.2.pdf}
	\end{figure}
	
	\newpage
	
	Plot 2:
	
	\lstinputlisting[language=R, firstline=260, lastline=278]{PS03_SVU.R}
	\lstinputlisting[language=R, firstline=305, lastline=307]{PS03_SVU.R}	
	
	\begin{figure}[h!]
		\caption{Voting Turnout Lower Population Provinces, NA's \bf{not} included}\label{fig4}
		\centering
		\includegraphics[width=1\textwidth]{3.1.pdf}
	\end{figure}
	
	\newpage
	
	Plot 3:
	
	\lstinputlisting[language=R, firstline=238, lastline=257]{PS03_SVU.R}
	\lstinputlisting[language=R, firstline=316, lastline=317]{PS03_SVU.R}
	
	\begin{figure}[h!]
		\caption{Voting Turnout Higher Population Provinces, NA's included}\label{fig5}
		\centering
		\includegraphics[width=1\textwidth]{3.4.pdf}
	\end{figure}
	\newpage
	
	Plot 4:
	
	\lstinputlisting[language=R, firstline=281, lastline=301]{PS03_SVU.R}
	\lstinputlisting[language=R, firstline=313, lastline=314]{PS03_SVU.R}
	
	\begin{figure}[h!]
		\caption{Voting Turnout Higher Population Provinces, NA's \bf{not} included}\label{fig6}
		\centering
		\includegraphics[width=1\textwidth]{3.3.pdf}
	\end{figure}
	
	\newpage
	
	\item Create your own reusable custom theme. Apply your theme to one of the previous plots and add:
	\begin{enumerate}
		\item An improved title summarizing the main substantive takeaway.
		\item A more informative subtitle describing the sample and variables.
		\item A caption noting data source, weighting, and key coding decisions.
		\item At least one direct annotation using \texttt{ggrepel} that calls out a key pattern.
	\end{enumerate}

	\lstinputlisting[language=R, firstline=47, lastline=82]{PS03_SVU.R}


\begin{figure}[h!]
	\caption{Ideological Self-Placement by Party Affiliation with Custom Theme}\label{fig7}
	\centering
	\includegraphics[width=1\textwidth]{plot_2.4.pdf}
\end{figure}


\end{enumerate}


\end{document}
